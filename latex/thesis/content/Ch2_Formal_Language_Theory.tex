\chapter{\rm\bfseries Formal Language Theory}
\label{ch:chapter01}

In computer science, it is common to conflate two distinct notions for a set. The first is a collection sitting on some storage device, e.g., a dataset. The second is a lazy construction: not an explicit collection of objects, but a representation that allows us to efficiently determine membership on demand. This lets us represent infinite sets without requiring an infinite amount of storage. Inclusion then, instead of being simply a lookup query, becomes a decision procedure. This is the basis of formal language theory.

The representation we are chiefly interested in is called a \textit{grammar}, a common metanotation for specifying the syntactic constraints on programs, shared by nearly every programming language. Programming language grammars are overapproximations to the true language, but provide a reasonably detailed specification for rejecting invalid programs and parsing valid ones.

Formal languages are arranged in a hierarchy of containment, where each language family strictly contains its predecessors. On the lowest level of the hierarchy are finite languages. Type 3 contains finite and infinite languages generated by a regular grammar. Type 2 contains context-free languages, which admit parenthetical nesting. Supersets, such as the recursively enumerable sets, are Type 0. There are other kinds of formal languages, such as logics and circuits, which are incomparable with the Chomsky hierarchy.

Most programming languages leave level 2 after the parsing stage, and enter the realm of type theory. At this point, compiler authors layer additional semantic refinements on top of syntax, but must deal with phase ordering problems related to the sequencing of such analyzers, breaking commutativity and posing challenges for parallelization. This lack of compositionality is a major obstacle to the development of modular static analyses.

The advantage of dealing with formal language representations is that we can reason about them algebraically. Consider the context-free grammar: the arrow $\rightarrow$ becomes an $=$ sign, $\mid$ becomes $+$ and $AB$ becomes $A \times B$. The ambiguous Dyck grammar, then, can be seen as a system of equations.

\begin{equation}
    S \rightarrow ( ) \mid ( S ) \mid S S \Longleftrightarrow f(x) = x^2 + x^2 f(x) + f(x)^2
\end{equation}

\noindent We will now solve for $f(x)$, giving us the generating function for the language:

\begin{equation}
  0 = f(x)^2 + x^2 f(x) - f(x) + x^2
\end{equation}

\noindent Now, using the quadratic equation, where $a = 1, b = x^2 - 1, c = x^2$, we have:

\begin{equation}
  f(x) = \frac{-b \pm \sqrt{b^2 - 4ac}}{2a} = \frac{-x^2 + 1 \pm \sqrt{x^4 - 6x^2 + 1}}{2}
\end{equation}

\noindent Note there are two solutions, but only one where $\lim_{x\rightarrow 0} = 1$. From the ordinary generating function (OGF), we also have that $f(x)=\sum _{n=0}^{\infty }f_nx^{n}$. Expanding $\sqrt{x^4 - 6x^2 + 1}$ via the generalized binomial theorem, we have:

\begin{align}
f(x) = (1+u)^{\alpha }&=\sum _{k=0}^{\infty }\;{\binom {\alpha }{k}}\;u^{k}\\
  &=\sum _{k=0}^{\infty }\;{\binom {\frac{1}{2} }{k}}\;(x^4 - 6x^2)^{k} \text{ where } u = x^4-6x^2
\end{align}

Now, to obtain the number of ambiguous Dyck trees of size $n$, we can extract the $x^n$-th coefficient using the binomial series:

\begin{align}
  [x^n]f(x) &= [x^n]\frac{-x^2 + 1}{2} + \frac{1}{2}[x^n]\sum _{k=0}^{\infty }\;{\binom {\frac{1}{2} }{k}}\;(x^4 - 6x^2)^{k}\\
  [x^n]f(x) &= \frac{1}{2}{\binom {\frac{1}{2} }{n}}\;[x^n](x^4 - 6x^2)^n = \frac{1}{2}{\binom {\frac{1}{2} }{n}}\;[x^n](x^2 - 6x)^n
\end{align}

We can use this technique, first described by Flajolet \& Sedgewick~\cite{flajolet2009analytic}, to count the number of trees of a given size or distinct words in an unambiguous CFG. This lets us understand grammars as a kind of algebra, which is useful for enumerative combinatorics on words and syntax-guided synthesis.

Naturally, like algebra, there is also a kind of calculus to formal languages. Janusz Brzozowski~\cite{brzozowski1964derivatives} introduced the derivative operator for regular languages, which can be used to determine membership, and extract subwords from the language. This operator has been extended to CFLs by Might et al.~\cite{might2011parsing}, and is the basis for a family of elegant parsing algorithms.

The Brzozowski derivative has an extensional and intensional form. Extensionally, we have $\partial_a L = \{b \in \Sigma^* \mid ab \in L\}$. Intensionally, we have an induction over generalized regular expressions (GREs), which are a superset of regular expressions that also support intersection and negation.\vspace{-1cm}

\begin{multicols}{2}
  \begin{eqnarray*}
    \phantom{-}\partial_a( & \varnothing & )= \varnothing                                           \\
    \phantom{-}\partial_a( & \varepsilon & )= \varnothing                                           \\
    \phantom{-}\partial_a( & a           & )= \varepsilon                                           \\
    \phantom{-}\partial_a( & b           & )= \varnothing  \text{ for each } a \neq b               \\
    \phantom{-}\partial_a( & R^*         & )= (\partial_x R)\cdot R^*                               \\
    \phantom{-}\partial_a( & \neg R      & )= \neg \partial_a R                                     \\
    \phantom{-}\partial_a( & R\cdot S    & )= (\partial_a R)\cdot S \vee \delta(R)\cdot\partial_a S \\
    \phantom{-}\partial_a( & R\vee S     & )= \partial_a R \vee \partial_a S                        \\
    \phantom{-}\partial_a( & R\land S    & )= \partial_a R \land \partial_a S
  \end{eqnarray*} \break\vspace{-0.5cm}
  \begin{eqnarray*}
    \phantom{---}\delta(& \varnothing &)= \varnothing                                      \\
    \phantom{---}\delta(& \varepsilon &)= \varepsilon                                      \\
    \phantom{---}\delta(& a           &)= \varnothing                                      \\
    \phantom{---}\delta(& R^*         &)= \varepsilon                                      \\
    \phantom{---}\delta(& \neg R      &)= \varepsilon \text{ if } \delta(R) = \varnothing  \\
    \phantom{---}\delta(& \neg R      &)= \varnothing \text{ if } \delta(R) = \varepsilon  \\
    \phantom{---}\delta(& R\cdot S    &)= \delta(R) \land \delta(S)                        \\
    \phantom{---}\delta(& R\vee S     &)= \delta(R) \vee  \delta(S)                        \\
    \phantom{---}\delta(& R\land S    &)= \delta(R) \land \delta(S)
  \end{eqnarray*}
\end{multicols}

Similar to sets, it is possible to combine languages by manipulating their grammars, mirroring the setwise notions of union, intersection, complementation and difference over languages. These operations are convenient for combining, for example, syntactic and semantic constraints on programs. For example, we might have two grammars, $G_a, G_b$ representing two properties that are desirable or necessary for a program to be considered valid. %We can treat valid programs $P$ as a subset of the language intersection $P \subseteq \mathcal{L}(G_a) \cap \mathcal{L}(G_b)$.

Like all representations, grammars are themselves a trade-off between expressiveness and efficiency. It is possible to represent the same finite set with multiple representations of varying complexity. For example, the set of strings containing ten or fewer balanced parentheses can be expressed as a finite automaton containing millions of states, or a simple language conjunction containing a few productions, e.g., $\mathcal{L}\Big(S \rightarrow ( ) \mid (S) \mid S S \Big) \cap \Sigma^{[0,10]}$.

The choice of representation is heavily usage-dependent. For example, if we are interested in recognition, we might favor a disjoint representation, allowing properties to be checked independently without merging, whereas if we are interested in generation or deciding non-emptiness, we might prefer a unified representation which can be efficiently sampled without rejection.

Union, concatenation and repetition are all mundane in the theory of formal languages. Intersection and negation are more challenging concepts to borrow from set theory, and do not translate naturally into the Chomsky hierarchy. For example, the intersection of two CFLs is Turing Complete, but the intersection of a CFL and a regular language is a CFL.

Deciding intersection non-emptiness (INE) of a finite collection of finite automata is known to be PSPACE-complete~\cite{kozen1977lower}. It is still unknown whether a faster algorithm than the product construction exists for deciding INE of just two finite automata.

The textbook algorithm proceeds as follows: create an automaton containing the cross-product of states, and simulate both automata in lockstep, creating arcs between states that are co-reachable. If a final state is reachable in the product automaton, then the intersection is non-empty. This requires space proportional to the Cartesian product of the two states.

\begin{figure}[h]
  \caption{TODO: depict product construction for finite automata here.}
  \end{figure}

The goal of this thesis is to speed up the product construction by leveraging (1) parameterized complexity (2) pruning and (3) parallelization to speed up the wallclock runtime of the product construction and generalize it to CFG-REG intersections. We show it is possible to decide INE in realtime for Levenshtein automata and build a tool to demonstrate it on real-world programming languages and grammars.

Finally, we show a probabilistic extension of the REG-CFL product construction, which can be used to decode the top-K most probable words in the intersection of two languages. This is useful for applications in natural language processing, where we might want to find the most natural word that satisfies multiple constraints, such as being a valid repair with fewer than $k$ edits whose probability is maximized.

\clearpage