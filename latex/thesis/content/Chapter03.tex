\chapter{\rm\bfseries Probabilistic Program Repair}
\label{ch:chapter03}

As we have seen, the problem of program repair is highly underdetermined. To resolve this ambiguity, we will use a probabilistic model to induce a distribution over the language of valid programs. This distribution will guide the repair process by assigning a likelihood to each possible repair. Then, taking the maximum over all possible repairs, we can find the most likely repair consistent with the constraints and the observed program.

Specifically, we will define an ordering over strings by their likelihood under the probabilistic model. We then define a repair as the most likely string consistent with the observed program and the grammar. We factorize the probability of a string as the product of the probability of each token in the string, conditioned on the previous tokens. This allows us to compute the likelihood of a string in a left-to-right fashion.

This probabilistic model will generally admit programs that are locally probable, but globally inconsistent with the grammar. To enforce syntactic validity, we will use the probabilistic language model to ``steer'' a generative sampler through the automaton representing the repair language. This has two advantages: first, it allows us to sample from the repair language incrementally, and second, it ensures that subsequences with high probability are retrieved first, and all trajectories are syntactically valid.