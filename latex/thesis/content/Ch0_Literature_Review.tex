\chapter{\rm\bfseries Related Literature}
\label{ch:litreview}

Translating ideas into computer programs demands a high degree of precision, as computers have strict criteria for admitting valid programs. These constraints act as a failsafe against faulty programs and runtime errors, but can be tedious to debug. During the editing process, constraints are invariably violated by the hasty or inexperienced programmer, requiring manual repair. To assist with this task, automated program repair (APR) attempts to generate possible revisions from which the author may choose. This subject has been closely investigated by programming language research and treated in a number of existing literature reviews~\cite{monperrus2018living, le2021automatic}. We direct our attention primarily towards syntax repair, which attempts to fix parsing errors, the earliest stage in program analysis.

\section{Syntactic program repair}

Spellchecking is an early precursor to syntax repair that was originally developed for word processing and seeks to find, among a finite dictionary, the most likely intended revision of a misspelled word~\cite{kernighan1990spelling}. Similarly, syntax repair considers the case where this dictionary is not necessarily finite, but rather generated by a grammar representing a potentially infinite collection of words called a \textit{language}. This has applications in natural language processing~\cite{bryant2023grammatical}, although we are primarily interested in programming languages. In the case of programming language syntax, the language and corresponding grammar is typically context-free~\cite{chomsky1959algebraic}.

Various methods have been proposed to handle syntactic program errors, which have been a longstanding open problem since the advent of context-free languages. In 1972, Aho and Peterson~\cite{aho1972minimum} first introduce an algorithm that returns a syntactically valid sequence whose distance from the original sequence is minimal. Their method guarantees that a valid repair will be found, but only generates a single repair and does attempt to optimize the naturalness of the generated solution, only the proximity and validity.

While algorithmically elegant, deterministic repair methods lack the flexibility to model the natural features of source code. It does not suffice to merely suggest parseable repairs, but a pragmatic solution must also generate suggestions a human is likely to write in practice. To model code conventions, stylistic patterns and other programming idioms that are not captured in the formal grammar, researchers have adopted techniques from natural language processing, in particular recent advances in neural language modeling.

Recent work attempts to use neural language models to generate probable fixes. For example, Yasunaga et al.~\cite{yasunaga2021break} use an unsupervised method to synthetically corrupt natural source code (simulating a typographic noise process), then learn a second model to repair the broken code, using the uncorrupted source as the ground truth. This method does not require a parallel corpus of source code errors and fixes, but can produce a misaligned noise model and fail to generalize to out-of-distribution samples. It also does not guarantee the generated fix is valid.

Sakkas et al.~\cite{sakkas2022seq2parse} introduce a neurosymbolic model, Seq2Parse, which adapts the Early parser~\cite{earley1970efficient} with a learned PCFG and a transformer-classifier to predict error production rules. This approach aims to generate only sound repairs, but lacks the ability to generate every valid repair within a given edit distance. While this has the benefit of better interpretability than end-to-end neural repair models, it is not clear how to scale up this technique to handle additional test-time compute.

Neural language models are adept at learning statistical patterns, but often sacrifice validity, precision or latency. Existing neural repair models are prone to misgeneralize and hallucinate syntactically invalid repairs and do not attempt to sample from the space of all and only valid repairs. As a consequence, they have difficulty with inference scaling, where additional test time compute does not translate to improved precision on the target domain. Furthermore, the generated samples may not even be syntactically valid, as we observe in practice.

Our work aims to address all of these concerns. We try to generate every nearby valid program and prioritize the solutions by naturalness, while ensuring response time is tolerable. In other words, we attempt to satisfy soundness, completeness, naturalness and latency simultaneously.

\section{Semantic program repair}

Automated program repair is either implicitly or explicitly defined over a \textit{search space}, which is the space of all possible solutions. Previously, we looked at a very coarse-grained approximation, based on syntactic validity. In practice, we might layer additional refinements on top of these syntactic constraints, corresponding to so-called \textit{semantic} properties such as type-soundness or well-formedness. This additional criteria lets us \textit{prune} invalid solutions or \textit{quotient} the search space by an equivalence relation, often vastly reducing the set of candidate repairs.

Semantically valid program representations are typically framed as a subset of the syntactically valid ones. In some cases, the syntax of a programming language is not even context-free, in which case syntax repair may be viewed as a kind of semantic repair. The C/C++~\cite{mcpeak2004elkhound} language, for example, implements a so-called lexer-hack, introducing type names into the symbol table which is used for parsing. Though generally considered in poor taste from a language design perspective, handling these kinds of scenarios is important for building tooling for practical programming languages.

Methods such as angelic program repair have been developed to synthesize whole programs from a library of existing components. Shi et al.'s FrAngel~\cite{shi2019frangel} is one such example, which uses component-based program synthesis in conjunction with angelic nondeterminism to repair a broken program. The idea of angelic execution can be retraced to Bod\'ik et al.~\cite{bodik2010programming} who attributes the original idea to Floyd's nondeterministic \texttt{choice} operator. In the context of semantic program repair, angelic execution has been successfully developed for program synthesis by Singh et al.~\cite{singh2013automated} for auto-grading and providing feedback on programming assignments.

The idea of angelic execution has also been employed to great effect to assist with automated program repair. In particular, Long \& Rinard~\cite{long2016automatic} introduce a tool called Prophet and use a very similar evaluation to the one we propose to generate and ranks candidate patches from a search space, then rank the generated patches according to a learned probabilistic model. Chandra et al.~\cite{chandra2011angelic} also use angelic execution to guide the search for semantic program repairs. Both systems bypass the syntactic checking stage and search directly for semantic repairs, using a set of test cases. They do not use an explicit search space, but run the tests to reject invalid candidates. Their approach is closely related to fault localization and mutation testing in the software engineering literature.

\clearpage