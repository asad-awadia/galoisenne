\chapter{\rm\bfseries Related Literature}
\label{ch:litreview}

Translating ideas into computer programs demands a high degree of precision, as computers have strict criteria for admitting valid programs. These constraints act as a failsafe against faulty programs and runtime errors, but can be tedious to debug. During the editing process, these constraints are invariably violated by the hasty or inexperienced programmer, requiring manual repair. To assist with this task, automated program repair (APR) attempts to generate possible revisions from which the author may choose. This subject has been closely investigated by programming language research and treated in a number of existing literature reviews~\cite{monperrus2018living, le2021automatic}. We direct our attention primarily towards syntax repair, which attempts to fix parsing errors, the earliest stage in program analysis.

\section{Syntax Repair}

Spellchecking is an early precursor to syntax repair was originally developed for word processing and seeks to find, among a finite dictionary, the most likely intended revision of a misspelled word~\cite{kernighan1990spelling}. Similarly, syntax repair considers the case where this dictionary is not necessarily finite, but rather generated by a grammar representing a potentially infinite collection of words called a \textit{language}. This has applications in natural language processing~\cite{bryant2023grammatical}, although we are primarily interested in programming languages. In the case of programming language syntax, the language and corresponding grammar is typically context-free~\cite{chomsky1959algebraic}.

Various methods have been proposed to handle syntactic program errors, which have been a longstanding open problem since the advent of context-free languages. In 1972, Aho and Peterson~\cite{aho1972minimum} first introduce an algorithm that returns a syntactically valid sequence whose distance from the original sequence is minimal. Their method guarantees that a valid repair will be found, but only generates a single repair and does attempt to optimize the naturalness of the generated solution, only the proximity and validity.

While algorithmically elegant, deterministic repair methods lack the flexibility to model the natural features of source code. It does not suffice to merely suggest parseable repairs, but a pragmatic solution must also generate suggestions a human is likely to write in practice. To model code conventions, stylistic patterns and other programming idioms that are not captured in the formal grammar, researchers have adopted techniques from natural language processing, in particular recent advances in neural language modeling.

Recent work attempts to use neural language models to generate probable fixes. For example, Yasunaga et al.~\cite{yasunaga2021break} use an unsupervised method to synthetically corrupt natural source code, simulating a typographic noise process, then learn a second model to repair the synthetically corrupted broken code, using the original source code as the ground truth. This method does not require a parallel corpus of broken and fixed source code, but can learn a misaligned noise model and fail to generalize to out-of-distribution samples. It also does not guarantee the generated fix is valid.

Sakkas et al.~\cite{sakkas2022seq2parse} introduce a neurosymbolic model, Seq2Parse, which adapts the Early parser~\cite{earley1970efficient} with a learned PCFG and a transformer-based classifier to predict error production rules. This approach aims to generate only sound repairs, but lacks the ability to generate every valid repair within a given edit distance. It has the benefit of better interpretability than end-to-end neural repair models, but is not clear how to scale up the technique to handle additional test-time compute.

Neural language models are adept at learning statistical patterns, but often sacrifice validity, precision or latency, being prone to misgeneralize and hallucinate syntactically invalid repairs. Existing neural repair models do not attempt to sample from the space of all and only valid repairs. As a consequence, they have difficulty with inference scaling, where additional test time compute does not translate to improved precision on the target domain. Furthermore, the generated samples may not even be syntactically valid, as we observe in practice.

Our work aims to address all of these concerns. We try to generate every nearby valid program and prioritize the solutions by naturalness, while ensuring response time is tolerable. In other words, we attempt to satisfy soundness, completeness, naturalness and latency simultaneously.


\clearpage